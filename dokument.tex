%╔════════════════════════════╗
%║	  Szablon dostosował	  ║
%║	mgr inż. Dawid Kotlarski  ║
%║		  06.10.2024		  ║
%╚════════════════════════════╝
\documentclass[12pt,twoside,a4paper,openany]{article}

    \input{preambula_pakiety.tex}
    \input{preambula_ustawienia.tex}

%definicja składni mikrotik
\usepackage{fancyvrb}
\DefineVerbatimEnvironment{MT}{Verbatim}%
{commandchars=\+\[\],fontsize=\small,formatcom=\color{red},frame=lines,baselinestretch=1,} 
\let\mt\verb 
%zakonczenie definicji składni mikrotik

\usepackage{fancyhdr}    %biblioteka do nagłówka i stopki

			
\begin{document}
   
    \renewcommand{\figurename}{Rys.}    %musi byc pod \begin{document}, bo w~tym miejscu pakiet 'babel' narzuca swoje ustawienia
    \renewcommand{\tablename}{Tab.}     %j.w.
    \thispagestyle{empty}               %na tej stronie: brak numeru
 
    \begin{center}
        \scriptsize
        \renewcommand{\arraystretch}{1.35}
        % \setlength{\tabcolsep}{7.7pt}
        \setlength{\tabcolsep}{25pt}
        \begin{tabular}{|c|c|c|}
            \hline %chktex 44
            \multicolumn{3}{|c|}{Wydział Nauk Inżynieryjnych ANS w Nowym Sączu}   \\
            \multicolumn{3}{|c|}{Inżynieria oprogramowania – laboratorium}   \\
            \hline %chktex 44
            \multicolumn{2}{|l|}{Temat: Diagram przypadków użycia.} & Symbol \\
            \multicolumn{2}{|l|}{} & IO\_L1 \\
            \hline %chktex 44
            Nazwisko i imię: & Ocena sprawozdania & Zaliczenie:\\
            \cline{2-3}
            Placeholder & & \\
            \hline
            Data wykonania ćwiczenia:&\multicolumn{2}{l|}{Grupa:}\\
            \today &\multicolumn{2}{l|}{L3}\\
            \hline
        \end{tabular}
    \end{center}


 \pagestyle{fancy}

    % \newpage

    %formatowanie spisu treści i~nagłówków
    \renewcommand{\cftbeforesecskip}{8pt}
    \renewcommand{\cftsecafterpnum}{\vskip 8pt}
    \renewcommand{\cftparskip}{3pt}
    \renewcommand{\cfttoctitlefont}{\Large\bfseries\sffamily}
    \renewcommand{\cftsecfont}{\bfseries\sffamily}
    \renewcommand{\cftsubsecfont}{\sffamily}
    \renewcommand{\cftsubsubsecfont}{\sffamily}
    \renewcommand{\cftparafont}{\sffamily}
    %koniec formatowania spisu treści i nagłówków
     
    % \tableofcontents    %spis treści
    \thispagestyle{plain}
    
    % \newpage

    
%%%%%%%%%%%%%%%%%%% treść główna dokumentu %%%%%%%%%%%%%%%%%%%%%%%%%

   \section{Wstęp}
   \clearpage
   \thispagestyle{fancy}
   \section{Wstęp}
%    \input{tex/rozdzial2.tex}
%    \input{tex/rozdzial3.tex}
%    \input{tex/rozdzial4.tex}
%    \input{tex/rozdzial5.tex}
   
       
%%%%%%%%%%%%%%%%%%% koniec treść główna dokumentu %%%%%%%%%%%%%%%%%%%%%
	% \newpage
% \addcontentsline{toc}{section}{Literatura}
% Modified by: Maciej Wójs 
%Wyświetlanie bibliografii jako "Literatura", w spisie treści wyświetla się podbnie do \section
% \printbibliography[heading=bibnumbered, label=Literatura, title=Literatura]

    \newpage
    \hypersetup{linkcolor=black}
    \renewcommand{\cftparskip}{3pt}
    % \clearpage
    \renewcommand{\cftloftitlefont}{\Large\bfseries\sffamily}
    % \listoffigures
    % \addcontentsline{toc}{section}{Spis rysunków}
	% \thispagestyle{fancy}
	% 
    % \newpage
    % \renewcommand{\cftlottitlefont}{\Large\bfseries\sffamily}
    % \def\listtablename{Spis tabel}
    % \addcontentsline{toc}{section}{Spis tabel}\listoftables 
	% \thispagestyle{fancy}
	
	% \newpage
	% \renewcommand{\cftlottitlefont}{\Large\bfseries\sffamily}
	% \renewcommand\lstlistlistingname{Spis listingów}
	% \addcontentsline{toc}{section}{Spis listingów}\lstlistoflistings 
	% \thispagestyle{fancy}
	


    %lista rzeczy do zrobienia: wypisuje na koñcu dokumentu, patrz: pakiet todo.sty
    \todos
    %koniec listy rzeczy do zrobienia
\end{document}
